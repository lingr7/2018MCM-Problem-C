\section{Conclusions}
The energy configuration analysis we established is shown in \autoref{fig:mianjitu} as an area chart after integrat variables. The energy structure of each state has its own characteristics.In Arizona, fossil fuels have always dominated, still accounting for well over half in 2009. In Texas, the proportion of biomass and wind energy is very low.\\
Through Energy Proportion Model, we get the proportion of fossil fuels
energy, renewable energy and nuclear energy in each state over time as shown in \autoref{fig:fenbutu}.
We define the proportion of energy $ P(t) $. Similarly we obtain Growth Rate Model defining the proportion of energy  $ v(t) $.\\We choose industry and  population to discuss influential Factors of the similarities and differences about usage of cleaner, renewable energy sources between the four states. The similarity is that the ratio of impact factors of AZ, NM, TX called as $ Q_{1}(t) $, $ Q_{2}(t) $, $ Q_{3}(t) $ is $ 3.412:1:1 $.
The difference is that the ratio of impact factors of CA states changes over time, with the attached illustrations of CA shown in \autoref{fig:Industry}.\\
The population of four states increases approximately linearly over time. However, the disaggregation analysis structure based on the population scale model shows that the energy consumption growth in each state has great volatility in growth of population size.\\
We use analytical hierarchy process, abbreviated to AHP. Determine three indexes we choose-average price, import expenditure size, size of productivity of electricity as the evaluation criteria weights.The largest total sort weight is in TX whose total sort weight is 0.4043. In 2009, Texas had the "best" have the “best” profile for use of cleaner, renewable energy.\\
According to $ P(t) $ and  $ v(t) $ based on our models, we get the predictive value of the energy profile of each state for 2025 and 2050 as shown in \autoref{tab:ProPredict} and \autoref{tab:GroPredict}.\\
The average price of the four states in 2025 should be set at about 28.26(80\% of the predictive values absent any policy changes) dollars.\\
In 2050 the average price should be set at about 34.97(80\% of the predictive values absent any policy changes) dollars. We state it as the goal for this new four-state energy compact.
We discuss three actions the four states might take to meet their energy
compact goals.
\begin{itemize}
	\item Support clean coal technology.
    \item Support photovoltaic solar cell semiconductor materials research.
    \item Apply preferential policies to energy transactions between the states.
\end{itemize}

