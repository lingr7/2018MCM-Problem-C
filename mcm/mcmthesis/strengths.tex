\section{Strengths and weaknesses}
\subsection{Strength}
\begin{itemize}
	\item We use two models to predict the consumption of various energy sources in 2025 and 2050 from both the proportion and the growth rate of energy consumption. It  make up for the errors caused by some uncertain factors. We take the industrial development over time into account. At the same time, we consider that the consumption of various forms of energy will have a certain periodicity and volatility due to the influence of industry, environment, population, technology and other factors.
	\item The growth rate model adopts the method of moving average to define $ c(t) $, which eliminates some uncontrollable factors such as random errors of data records and the influence of the previous year on the following year.
	\item The population size decomposition model called IPAT quantifies the impact of population on clean energy.
\end{itemize}
\subsection{Weakness}
\begin{itemize}
	\item Since the predictive value of Proportion Model depends on the total energy consumption, the linear fit of the total energy consumption less prime in the consideration of factors. It will make all kinds of energy consumption too large.	
	\item The two models in CA predict a larger error in the share of renewable energy in 2050.
	\item No sensitivity analysis was done in the evaluation Evaluation Criteria 
	\item Incomplete assumptions. There are claims that natural gas reserves in 2040 depletion.Our model does not consider the limited reserves.
	\item In this paper, the solution of the model is limited to the capacity of the computer, and it can't achieve higher accuracy.
	\item Our model cannot resolve the existence of information cascade (the most influential one is not always the best one) phenomena.
\end{itemize}

